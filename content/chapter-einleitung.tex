\chapter{Einleitung}
Seit einigen Jahren findet man immer häufiger Bedienungen mit Touchscreens. Ihr Einsatzbereich scheint keine Grenzen zu haben und so hat sich die Technologie in den letzten Jahren rasant weiterentwickelt.


\section{Ziel des Projekts}
In dieser Projektarbeit ist es Ziel, die Aufnahme und Auswertung der Koordinaten mittels eines resistiven Touchscreens. Zum Schluss soll die Steuerung noch dahingehend erweitert werden, dass damit die Maus eines PC's gesteuert werden kann.


\section{Grundlagen}
\subsection{Resistives Touchscreen}
Es gibt bei den resistiven Touchscreens zwei unterschiedliche Gruppen. Zum einen gibt es 4-Wire resistive Touchscreens. Diese haben vier Anschlüsse, über die die Positionsauswertung läuft. 
Zudem gibt es noch die 5-Wire resistive Touchscreens. Hier werden fünf Anschlüsse benötigt um eine Positionsauswertung durchführen zu können. 

Bei einem 4-Wire resistiven Touchscreen gibt es zwei Ebenen, bei denen die obere auf die untere gedrückt werden kann. 
Eine Ebene ist mit Elektronen geladen und über die andere Ebene misst man die Spannung, die bei der Berührung abfällt. 
Für die Andere Richtung ist es das gespiegelt. Die Ebene, die  die Spannung in die eine Richtung misst, ist für die andere Richtung mit Elektronen geladen. Die Messung der Spannung wird dann von der anderen Ebene durchgeführt.
Über die unterschiedliche Beträge der Spannung lässt sich dann die Koordinate in x-Richtung und y-Richtung bestimmen. 

Die 5-Wire resistive Touchscreens haben ebenfalls zwei Ebene. Der Unterschied zu den 4-Wire Touchscreen liegt darin, dass es nur eine Ebene gibt, die geladen ist. Normalerweise ist es die untere Schicht. Die obere Schicht misst für x-Richtung und y-Richtung die abfallende Spannung.
Auch hier wird durch Druck auf die obere Schicht ein Kontakt zwischen den beiden Ebenen erstellt, damit die Messung durchgeführt werden kann. 

Die 5-Wire Touchscreens sind gegenüber den 4-Wire Widerstandsfähiger und haben dadurch eine längere Lebenserwartung.\cite{5w4w}

In diesem Projekt wird ein 4-Wire resistiver Touchscreen von der Firma Fujitsu verwendet.

Der verwendete Touchscreen besitzt für jede Richtung (x und y) jeweils zwei Anschlüsse \figref{fig:4w}. In x, wie auch in y-Richtung sind jeweils 2 Widerstände eingezeichnet. 
Diese sollen den Widerstand auf der jeweiligen Ebene, über die die Spannung abfällt, darstellen.
Die Aufnahme der Werte für einen Koordinatenwert wird über das Prinzip des Spannungsteiler realisiert.
\begin{figure}
    \centering
    \includegraphics[width=0.6\linewidth]{fig/4-wire.jpg}
    \caption{Schemadarstellung eines 4-Wire resistiven Touchscreen}
    \label{fig:4w}
\end{figure}
\subsection{Arduino Leonardo Board}
Für dieses Projekt wird das Arduino Leonardo mit dem ATmega32u4 verwendet. 
Der Unterschied zu einem Arduino Uno Board liegen darin, dass das Leonardo Board die Möglichkeit hat, sich als Peripherie an einem PC an zu melden. 
Der Mikrocontroller arbeitet nicht mit einem USB-Chip, sonder verarbeitet intern die serielle eingehende Daten und konvertiert diese für die Nutzung der USB-Schnittstelle.

